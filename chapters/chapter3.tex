\chapter{Chemistry and Thermodynamics}
\label{ch:microphysics}

\marginnote{\textbf{Suggested background reading:}
\begin{itemize}
\item \href{http://adsabs.harvard.edu/abs/2014arXiv1402.0867K}{Krumholz, M.~R. 2014, Phys.~Rep., 539, 49}, sections $3.1-3.2$ \nocite{krumholz14c}
\end{itemize}
\textbf{Suggested literature:}
\begin{itemize}
\item \href{http://adsabs.harvard.edu/abs/2010MNRAS.404....2G}{Glover, S.~C.~O., Federrath, C., Mac Low, M.-M., \& Klessen, R.~S. 2010, MNRAS, 404, 2} \nocite{glover10a}
\end{itemize}
}

Having completed our whirlwind tour of the observational phenomenology, we will now devote the next four chapters to understanding the physical processes that govern the behavior of the star-forming ISM and its transformation into stars. The goal here is to develop physical intuition for how this gas behaves, and to develop some analytic tools that we can use through the remainder of the course. This part begins with a discussion of the microphysics of the cold ISM.

\section{Chemical Processes in the Cold ISM}

We will begin our discussion of the microphysics of the cold ISM with the goal of understanding something important that should be clear from the observational discussion: the parts of the ISM associated with star formation are overwhelmingly molecular gas. This is in contrast to the bulk of the ISM, at least in the Milky Way and similar galaxies, where the bulk of interstellar matter is composed of atomic or ionized gas with few or no molecules. So why does the ISM in some places turn molecular, and how is this transition associated with star formation? We will focus this discussion on the most important atoms / molecules in the ISM: hydrogen / H$_2$ and carbon / oxygen / CO.

\subsection{Hydrogen Chemistry}
\label{ssec:Hchemistry}

Molecular hydrogen is a lower energy state than atomic hydrogen, so an isolated box of hydrogen left for an infinite amount of time will eventually become predominantly molecular. In interstellar space, though, the atomic versus molecular fraction in a gas is determined by a balance between formation and destruction processes.

Atomic hydrogen can turn into molecular hydrogen in the gas phase, but this process is extremely slow. This is ultimately due to the symmetry of the hydrogen molecule. To form an H$_2$ molecule, two H atoms must collide and then undergo a radiative transition that removes enough energy to leave the resulting pair of atoms in a bound state. However, two H atoms that are both in the ground state constitute a symmetric system, as does an H$_2$ molecule in its ground state. Because both the initial and final states are symmetric, one can immediately show from symmetry considerations that the system cannot emit dipole radiation. Formally, in semi-classical theory, the rate of transitions from a starting state $\left\langle\psi_{2\mathrm{H}}\right|$ to a final state $\left|\psi_{\mathrm{H}_2}\right\rangle$ is proportional to a matrix element of the form
\begin{equation}
\left\langle\psi_{2\mathrm{H}}\right| \mathcal{E} \mathbf{r} \left|\psi_{\mathrm{H}_2}\right\rangle,
\end{equation}
where $\mathcal{E} \mathbf{r}$ is the dipole radiation operator. However, one can immediately see that if $\psi_{2\mathrm{H}}$ and $\psi_{\mathrm{H}_2}$ are both symmetric, then the inner product is anti-symmetric, and its integral over all space is therefore zero, yielding a transition rate of zero. Transitions are possible only if one continues to the next order of expansion of the radiation field (which, in quantum field theory, constitutes thinking about multi-photon processes), or if one if one considers either starting or final states there are not symmetric (say because one of the H atoms is in an excited state, or the final H$_2$ molecule is in an excited state). Neither of these routes leads to an appreciable transition rate either: multi-photon processes are suppressed compared to single-photon ones by high powers of the fine structure constant, and the lowest-lying energy states of the H$_2$ molecule are energetic enough that only a negligible fraction of collisions have enough energy to produce them.

Due to this limitation, the dominant formation process is instead formation on the surfaces of dust grains. In this case the excess energy released by forming the molecule is transferred into vibrations in the dust grain lattice, and there is no need for forbidden photon emission. The rate of H$_2$ formation by surface catalysis is given by
\begin{equation}
\frac{1}{2} S(T,T_{\rm gr}) \eta(T_{\rm gr}) n_{\rm gr} n_H \sigma_{\rm gr} v_H.
\end{equation}
Here $S$ is the probability that a hydrogen molecule that hits a dust grain will stick, which is a function of both the gas temperature and the grain temperature. $\eta$ is the probability that a grain which sticks will migrate across the grain surface and find another H atom before it is evaporated off the grain surface $n_{\rm gr}$ and $n_H$ are the number densities of grains and hydrogen atoms, $\sigma_{\rm gr}$ is the mean cross section for a dust grain, and $v_H$ is the thermal velocity of the hydrogen atoms.

The last three factors can be estimated reasonably well from observations of dust extinction and gas velocity dispersions, while the former two have to be determined by laboratory measurements and/or theoretical chemistry calculations. Since I am not a chemist, and the literature on this problem is large and heavily dominated by experimental atomic beam chemists, I will simply report the result: for conditions appropriate to the edges of giant molecular clouds, the formation rate is roughly
\begin{equation}
\mathcal{R} n n_H,
\end{equation}
where $n_H$ and $n$ are the number densities of H atoms and H nuclei (in atomic or molecular form), respectively, and $\mathcal{R}\approx 3\times 10^{-17}$ cm$^3$ s$^{-1}$ is the rate coefficient. It may be a factor of a few lower in warmer regions where the sticking probability is reduced. This is for Milky Way dust content. If we go to a galaxy with less dust, the rate coefficient is presumably reduced proportionally.

The reverse process, destruction, is mostly due to photo-destruction. As with H$_2$ formation, things are somewhat complicated by the symmetry of the H$_2$ system. The binding energy of H$_2$ in the ground state is only 4.5 eV, but this doesn't mean that 4.5 eV photons can destroy it. A reaction of the form
\begin{equation}
{\rm H_2} + h\nu \rightarrow {\rm H}+{\rm H}
\end{equation}
is forbidden by symmetry for exactly the same reason as its inverse. The reaction can only occur if the H$_2$ molecule is in an excited state that thus asymmetric (almost never the case at molecular cloud temperatures), or unless one of the H atoms is left in an excited state, which would require an energy of $14.5$ eV. Photons with an energy that high are not generally available, because they can ionize neutral hydrogen and thus all get absorbed before propagating very far.

Instead, the main H$_2$ destruction process proceeds in two stages. Hydrogen molecules have a series of excited electronic states with energies of $11.2-13.6$ eV (corresponding to $912-1100$ \AA) above the ground state, which produce absorption features known as the Lyman and Werner bands. Since these energies exceed the binding energy of the H$_2$ molecule (4.5 eV), absorptions into them undergo radiative decay to a ground electronic state that can be unbound. This happens roughly 10-15\% of the time, depending on exactly which excited state the absorption is into.

Photons in the LW energy range are produced by hot stars, and the Galaxy is saturated with them, which is why most of the Galaxy's volume is filled with atomic or ionized rather than molecular gas. (There are some galaxies that are mostly molecular, for reasons we will see in a moment.)

If $E^*_{\nu}$ is the number density of photons as a function of frequency $\nu$, then the destruction rate of H$_2$ is
\begin{equation}
\int n_{H_2} \sigma_{H_2,\nu} c E^*_{\nu} f_{\rm diss,\nu}\, d\nu,
\end{equation}
where $n_{\rm H_2}$ is the molecular hydrogen number density, $\sigma_{H_2,\nu}$ is the absorption cross-section at frequency $\nu$, and $f_{\rm diss,\nu}$ is the dissociation probability when a photon of frequency $\nu$ is absorbed. The expression inside the integral is just the number of hydrogen molecule targets times the cross-section per target times the number of photons times the relative velocities of the photons and molecules ($=c$) times the probability of dissociation per collision. The integral in frequency goes over the entire LW band, from $912-1100$ \AA.

To understand the circumstances under which H$_2$ can form, we can take a simple example. Suppose we have some cloud of gas, which we will treat as a uniform slab, which has a beam of UV radiation shining on its surface. The number density of hydrogen nuclei in the cloud is $n$, and the UV radiation field shining on the surface has a photon number density $E^*_0$. The photon flux is $F^* = c E^*_0$.

As a result of this radiation field, the outer parts of the cloud are atomic hydrogen. However, when a hydrogen molecule absorbs a photon and then re-emits that energy, the energy generally comes out in the form of multiple photons of lower energy, which are no longer able to excite resonant LW transitions. Thus photons are being absorbed as hydrogen forms, and the number of photons penetrating the cloud decreases as one moves further and further into it. Eventually the number of photons drops to near zero, and the gas becomes mostly molecular. This process is known as self-shielding.

We can get a rough estimate of when self-shielding is important by writing down two equations to describe this process. First, let us equate the rates of H$_2$ formation and destruction, i.e.\ assume the cloud is in chemical equilibrium. (This is generally true because the reaction rates go as $n^2$, so as long as turbulence produces high density regions, there will be places where the reaction occurs quite fast.) This gives
\begin{equation}
n_H n \mathcal{R} = \int n_{H_2} \sigma_{H_2,\nu} c E^*_{\nu} f_{\rm diss,\nu}\, d\nu
\approx f_{\rm diss}  \int n_{H_2} \sigma_{H_2,\nu} c E^*_{\nu}\, d\nu.
\end{equation}
In the second step we have made the approximation that $f_{\rm diss}$ is roughly frequency-independent, which is true, since it only varies by factors of less than order unity.

Second, let us write down the equation for photon conservation. This just says that the change in photon number density as we move into the cloud is given by the rate at which collisions with H$_2$ molecules remove photons.
\begin{equation}
\frac{dF^*_{\nu}}{dx}= c \frac{dE^*_{\nu}}{dx} = -n_{H_2} \sigma_{H_2,\nu} c E^*_{\nu}
\end{equation}
In principle there should be a creation term at lower frequencies, representing photons absorbed and re-emitted, but we're going to focus on the higher LW frequencies, where there is only photon removal. The term on the right hand side is just the collision rate we calculated before.

Now we can integrate the second equation over frequency over the LW band. This gives
\begin{equation}
\frac{dE^*}{dx} = -\int n_{H_2} \sigma_{H_2,\nu} E^*_{\nu}\, d\nu,
\end{equation}
where $E^*$ is the frequency-integrated photon number density. If we combine this equation with the chemical balance equation, we get
\begin{equation}
\frac{dE^*}{dx} = -\frac{n_H n \mathcal{R}}{c f_{\rm diss}}
\end{equation}
This just says that the rate at which photons are taken out of the beam is equal to the recombination rate, increased by a factor of $1/f_{\rm diss}$ because only $\sim 1$ in 10 absorptions actually have to be balanced by a recombination.

If we make the further approximation that the transition from atomic to molecular hydrogen is sharp, so that $n_H\approx n$ throughout the atomic layer, and we assume that $\mathcal{R}$ does not vary with position, then the equation is trivial to integrate. At any depth $x$ inside the slab,
\begin{equation}
E^*(x) = E^*_0 - \frac{n^2\mathcal{R}}{c f_{\rm diss}} x.
\end{equation}
The transition to molecular hydrogen occurs where $E^*$ reaches zero, which is at $x_{H_2}= c f_{\rm diss} E^*_0 / (n^2 \mathcal{R})$. The total column of atomic hydrogen is
\begin{equation}
N_H = nx_{H_2} = \frac{c f_{\rm diss} E^*_0}{n\mathcal{R}}
\end{equation}

It is helpful at this point to put in some numbers. In the Milky Way, the observed interstellar UV field is $E^*_0=7.5\times 10^{-4}$ LW photons cm$^{-3}$, and we'll take $n=100$ cm$^{-3}$ as a typical number density in a region where molecules might form.  Plugging these in with $f_{\rm diss}=0.1$ and $\mathcal{R}=3\times 10^{-17}$ cm$^{-3}$ s$^{-1}$ gives $N_H = 7.5\times 10^{20}$, or in terms of mass, a column of $\Sigma=8.4$ $\msun$ pc$^{-2}$. More precise calculations give numbers closer to $2\times 10^{20}$ cm$^{-2}$ for the depth of the shielding layer on one side of a GMC. (Of course a comparable column is required on the other side, too.) Every molecular cloud must be surrounded by an envelope of atomic gas with roughly this column density.

This has important implications. First, this means that molecular clouds with column densities of $100$ $\msun$ pc$^{-2}$ in molecules must have $\sim 10\%$ of their total mass in the form of an atomic shield around them. Second, it explains why most of the Milky Way's ISM in the solar vicinity is not molecular. In the regions outside of molecular clouds, the mean column density is a bit under $10^{21}$ cm$^{-2}$, so the required shielding column is comparable to the mean column density of the entire atomic disk. Only when the gas clumps together can molecular regions form.

This also explains why other galaxies which have higher column densities also have higher molecular fractions. To take an extreme example, the starburst galaxy Arp 220 has a surface density of a few $\times 10^4$ $\msun$ pc$^{-2}$ in its nucleus, and the molecular fraction there is at least 90\%, probably more.

\subsection{Carbon / Oxygen Chemistry}
\label{ssec:cochemistry}

H$_2$ is the dominant species in molecular regions, but it is very hard to observe directly for the reasons discussed in Chapter \ref{ch:obscold} -- the temperatures are too low for it to be excited. Moreover, as we will discuss shortly, H$_2$ is also not the dominant coolant for the same reason. Instead, that role falls to the CO molecule.

Why is CO so important? The main reason is abundances: the most abundant elements in the universe after H and He are O, C, and N, and CO is the simplest (and, under ISM conditions, most energetically favorable) molecule that can be made from them. Moreover, CO can be excited at very low temperatures because its mass is much greater than that of H$_2$, and its dipole moment is weak but non-zero. (A weak dipole moment lowers the energy of radiation emitted, which in turn lowers the temperature needed for excitation.)

Just as in the bulk of the ISM, hydrogen is mostly H, in the bulk of the ISM the oxygen is mostly O and the carbon is mostly C$^+$. It's C$^+$ rather than C because the ionization potential of carbon is less than that of hydrogen, and as a result it tends to be ionized by starlight. So how do we get from C$^+$ and O to CO?

The formation of CO is substantially different than that of H$_2$ in that it is dominated by gas-phase rather than grain-surface reactions. Since the temperatures in regions where this reaction is taking place tend to be low, the key processes involve ion-neutral reactions. As those who have taken the diffuse matter class will know (and those who have not yet will learn), these are important because the rate at which they occur is to good approximation independent of temperature, while neutral-neutral reactions.

There are two main pathways to CO. One passes through the OH molecule, and involves a reaction chain that looks like
\begin{eqnarray}
{\rm H}_2 + {\rm CR} & \rightarrow & {\rm H}_2^+ + e + {\rm CR} \\
{\rm H}_2^+ + {\rm H}_2 & \rightarrow & {\rm H}_3^+ + {\rm H} \\
{\rm H}_3^+ + {\rm O} & \rightarrow & {\rm OH}^+ + {\rm H}_2 \\
{\rm OH}^+ + {\rm H}_2 & \rightarrow & {\rm OH}_2^+ + {\rm H} \\
{\rm OH}_2^+ + e & \rightarrow & {\rm OH} + {\rm H} \\
{\rm C}^+ + {\rm OH} & \rightarrow & {\rm CO}^+ + {\rm H} \\
{\rm CO}^+ + {\rm H}_2 & \rightarrow & {\rm HCO}^+ + {\rm H} \\
{\rm HCO}^+ + e & \rightarrow & {\rm CO} + {\rm H}.
\end{eqnarray}
Here CR indicates cosmic ray. There are also a number of possible variants (e.g., the OH$_2^+$ could form OH$_3^+$ before proceeding to OH. The second main route is through the CH molecule, where reaction chains tend to follow the general pattern
\begin{eqnarray}
{\rm C}^+ + {\rm H}_2 & \rightarrow & {\rm CH}_2^+ + h\nu \\
{\rm CH}_2^+ + e & \rightarrow & {\rm CH} + {\rm H} \\
{\rm CH} + {\rm O} & \rightarrow & {\rm CO} + {\rm H}.
\end{eqnarray}
The rate at which the first reaction chain manufactures CO is limited by the supply of cosmic rays that initiate the production of H$_2^+$, while the rate at which the second reaction chain proceeds is limited by the rate of the final neutral-neutral reaction. Which chain dominates depends on the cosmic ray ionization rate, density, temperature, and similar details. Note that both of these reaction chains require the presence of H$_2$. 

CO is destroyed via radiative excitation followed by dissociation in essentially the same manner as H$_2$. The shielding process for CO is slightly different however. As with H$_2$, photons that dissociate CO can be absorbed both by dust grains and by CO molecules. However, due to the much lower abundance of CO compared to H$_2$, the balance between these two processes is quite different than it is for hydrogen, with dust shielding generally the more important of the two. Moreover, there is non-trivial overlap between the resonance lines of CO and those of H$_2$, and thus there can be cross-shielding of CO by H$_2$.

At this point the problem is sufficiently complex that one generally resorts to numerical modeling. The net result is that clouds tend to have a layered structure. In poorly-shielded regions where the FUV has not yet been attenuated, H~\textsc{i} and C$^+$ dominate. Further in, where the FUV has been partly attenuated, H$_2$ and C$^+$ dominate. Finally a transition to H$_2$ and CO as the dominant chemical states occurs at the center.

For typical Milky Way conditions, the result is that the gas will be mostly CO once the V-band extinction $A_V$ exceeds $1-2$ mag. This corresponds to a column density of a few $\times 10^{21}$ cm$^{-2}$, or $\sim 20$ $\msun$ pc${-2}$, for Milky Way dust. In comparison, recall that typical GMC column densities are $\sim 10^{22}$ cm$^{-2}$, or $\sim 100$ $\msun$ pc$^{-2}$. This means that there is a layer of gas where the hydrogen is mostly H$_2$ and the carbon is still C$^+$, but it constitutes no more than a few tens of percent of the mass. However, in galaxies with lower dust to gas ratios, the layer where H$_2$ dominates but the carbon is not yet mostly CO can be much larger.

\section{Thermodynamics of Molecular Gas}

Having discussed the chemistry of molecular gas, we now turn to the problem of its thermodynamics. What controls the temperature of molecular gas? We have already seen that observations imply temperatures that are extremely low, $\sim 10$ K or even a bit less. How are such cold temperatures achieved? To answer this question, we must investigate what processes heat and cool the molecular ISM.

\subsection{Heating Processes}

The dominant heating process in the atomic ISM is the grain photoelectric effect: photons from stars with energies of $\sim 8-13.6$ eV hit dust grains and eject fast electrons via the photoelectric effect. The fast electrons then thermalize and deposit their energy at heat in the gas. The rate per H nucleus at which this process deposits energy can be written approximately (we assert without justification -- see a general ISM textbook, such as \citealt{draine11a}) as
\begin{equation}
\Gamma_{\rm PE} \approx 4.0\times 10^{-26} \chi_{\rm FUV} Z_d' e^{-\tau_d}\mbox{ erg s}^{-1}
\end{equation}
where $\chi_{\rm FUV}$ is the intensity of the FUV radiation field scaled to its value in the Solar neighborhood, $Z'_d$ is the dust abundance scaled to the Solar neighborhood value, and $\tau_d$ is the dust optical depth to FUV photons. The result is, not surprisingly, proportional to the radiation field strength (and thus the number of photons available for heating), the dust abundance (and thus the number of targets for those photons), and the $e^{-\tau_d}$ factor by which the radiation field is attenuated.

At FUV wavelengths, typical dust opacities are $\kappa_d \approx 500$ cm$^2$ g$^{-1}$, so at a typical molecular cloud surface density $\Sigma\approx 50 - 100$ M$_\odot$ pc$^{-2}$, $\tau_d \approx 5-10$, and thus $e^{-\tau_d} \approx 10^{-3}$. Thus in the interiors of molecular clouds, photoelectric heating is strongly suppressed simply because the FUV photons cannot get in. Typical photoelectric heating rates are therefore of order a few $\times 10^{-29}$ erg s$^{-1}$ per H atom deep in cloud interiors, though they can obviously be much larger at cloud surfaces or in regions with stronger radiation fields.

We must therefore consider another heating process: cosmic rays. The great advantage of cosmic rays over FUV photons is that, because they are relativistic particles, they have much lower interaction cross sections, and thus are able to penetrate into regions where light cannot. The process of cosmic ray heating works as follows. The first step is the interaction of a cosmic ray with an electron, which knocks the electron off a molecule:
\begin{equation}
\mbox{CR}+\mbox{H}_2 \rightarrow \mbox{H}_2^+ + \mbox{e}^- + \mbox{CR}
\end{equation}
The free electron's energy depends only weakly on the CR's energy, and is typically $\sim 30$ eV.

The electron cannot easily transfer its energy to other particles in the gas directly, because its tiny mass guarantees that most collisions are elastic and transfer no energy to the impacted particle. However, the electron also has enough energy to ionize or dissociate other hydrogen molecules, which provides an inelastic reaction that can convert some of its 30 eV to heat. Secondary ionizations do indeed occur, but in this case almost all the energy goes into ionizing the molecule (15.4 eV), and the resulting electron has the same problem as the first one: it cannot effectively transfer energy to the much more massive protons.

Instead, there are a number of other channels that allow electrons to dump their energy into motion of protons, and the problem is deeply messy. The most up to date work on this is Goldsmith et al.~(2012, ApJ, 756, 157), and we can very briefly summarize it here. A free electron can turn its energy into heat through three channels. The first is dissociation heating, in which the electron strikes an H$_2$ molecule and dissociates it:
\begin{equation}
\mbox{e}^- + {\rm H}_2 \rightarrow 2{\rm H} + e^{-}.
\end{equation}
In this reaction any excess energy in the electron beyond what is needed to dissociate the molecule (4.5 eV) goes into kinetic energy of the two recoiling hydrogen atoms, and the atoms, since they are massive, can then efficiently share that energy with the rest of the gas. A second pathway is that an electron can hit a hydrogen molecule and excite it without dissociating it. The hydrogen molecule then collides with another hydrogen molecule and collisionally de-excites, and the excess energy again goes into recoil, where it is efficiently shared. The reaction is
\begin{eqnarray}
\mbox{e}^- + {\rm H}_2 & \rightarrow & {\rm H}_2^* + e^{-} \\
{\rm H}_2^* + {\rm H}_2 & \rightarrow & 2 {\rm H}_2.
\end{eqnarray}
Finally, there is chemical heating, in which the H$_2^+$ ion that is created by the cosmic ray undergoes chemical reactions with other molecules that release heat. There are a large number of possible exothermic reaction chains, for example
\begin{eqnarray}
{\rm H}_2^+ + {\rm H}_2 & \rightarrow & {\rm H}_3^+ + {\rm H} \\
{\rm H}_3^+ + {\rm CO} & \rightarrow & {\rm HCO}^+ + {\rm H}_2 \\
{\rm HCO}^+ + {\rm e}^- & \rightarrow & {\rm CO} + {\rm H}.
\end{eqnarray}
Each of these reactions is exothermic, and results in heavy ions recoiling at high speed that can efficiently share their energy via collisions. Computing the total energy release requires summing over all these possible reaction chains, which is why the problem is ugly. The final results is that the energy yield per primary cosmic ray ionization is in the range $\sim 13$ eV under typical molecular cloud conditions, but that it can be several eV higher or lower depending on the local density, electron abundance, and similar variables.

Combining this with the primary ionization rate for cosmic rays in the Milky Way, which is observationally-estimated to be about  $\sim 10^{-16}$ s$^{-1}$ per H nucleus in molecular clouds, this gives a total heating rate per H nucleus
\begin{equation}
\Gamma_{\rm CR} \sim 2\times 10^{-27}\mbox{ erg s}^{-1}.
\end{equation}
The heating rate per unit volume is $\Gamma_{\rm CR} n$, where $n$ is the number density of H nuclei ($=2\times$ the density of H molecules). This is sufficient that, in the interiors of molecular clouds, it generally dominates over the photoelectric heating rate.

\subsection{Cooling Processes}

In molecular clouds there are two main cooling processes: molecular lines and dust radiation. Dust can cool the gas efficiently because dust grains are solids, so they are thermal emitters. However, dust is only able to cool the gas if collisions between dust grains and hydrogen molecules occur often enough to keep them thermally well-coupled. Otherwise the grains cool off, but the gas stays hot. The density at which grains and gas become well-coupled is around $10^4-10^5$ cm$^{-3}$, which is higher than the typical density in a GMC, so we won't consider dust cooling further at this point. We'll return to it later when we discuss collapsing objects, where the densities do get high enough for dust cooling to be important.

The remaining cooling process is line emission, and by far the most important molecule for this purpose is CO, for the reasons stated earlier. The physics is fairly simple. CO molecules are excited by inelastic collisions with hydrogen molecules, and such collisions convert kinetic energy to potential energy within the molecule. If the molecule de-excites radiatively, and the resulting photon escapes the cloud, the cloud loses energy and cools.

Let us make a rough attempt to compute the cooling rate via this process. A diatomic molecule like CO can be excited rotationally, vibrationally, or electronically. At the low temperatures found in molecular clouds, usually only the rotational levels are important. These are characterized by an angular momentum quantum number $J$, and each level $J$ has a single allowed radiative transition to level $J-1$. Larger $\Delta J$ transitions are strongly suppressed because they require emission of multiple photons to conserve angular momentum.

Unfortunately the CO cooling rate is quite difficult to calculate, because the lower CO lines are all optically thick. A photon emitted from a CO molecule in the $J=1$ state is likely to be absorbed by another one in the $J=0$ state before it escapes the cloud, and if this happens that emission just moves energy around within the cloud and provides no net cooling. The cooling rate is therefore a complicated function of position within the cloud -- near the surface the photons are much more likely to escape, so the cooling rate is much higher than deep in the interior. The velocity dispersion of the cloud also plays a role, since large velocity dispersions Doppler shift the emission over a wider range of frequencies, reducing the probability that any given photon will be resonantly re-absorbed before escaping.

In practice this means that CO cooling rates usually have to be computed numerically, and will depend on the cloud geometry if we want accuracy to better than a factor of $\sim 2$. However, we can get a rough idea of the cooling rate from some general considerations. The high $J$ levels of CO are optically thin, since there are few CO molecules in the $J-1$ states capable of absorbing them, so photons they emit can escape from anywhere within the cloud.
However, the temperatures required to excite these levels are generally high compared to those found in molecular clouds, so there are few molecules in them, and thus the line emission is weak. Moreover, the high $J$ levels also have high critical densities, so they tend to be sub-thermally populated, further weakening the emission.

On other hand, low $J$ levels of CO are the most highly populated, and thus have the highest optical depths. Molecules in these levels produce cooling only if they are within one optical depth the cloud surface. Since this restricts cooling to a small fraction of the cloud volume (typical CO optical depths are many tens for the $1\rightarrow 0$ line), this strongly suppresses cooling.

The net effect of combining the suppression of low $J$ transitions by optical depth effects and of high $J$ transitions by excitation effects is that cooling tends to be dominated a the single line produced by the lowest $J$ level for which the line is not optically thick. This line is marginally optically thin, but is kept close to LTE by the interaction of lower levels with the radiation field. Which line this is depends on the column density and velocity dispersion of the cloud, but typical peak $J$ values in Milky Way-like galaxies range from $J=2\rightarrow 1$ to $J=5\rightarrow 4$.

For an optically thin transition of a quantum rotor where the population is in LTE, the rate of energy emission per H nucleus from transitions between angular momentum quantum numbers $J$ and $J-1$ is given by
\begin{eqnarray}
\label{eq:lambdaco}
\Lambda_{J,J-1} & = & x_{\rm em} \frac{(2J+1)e^{-E_J/k_B T}}{Z(T)} A_{J,J-1} (E_J - E_{J-1}) \\
E_J & = & h B J (J+1) \\
A_{J,J-1} & = & \frac{512\pi^4 B^3\mu^2}{3hc^3} \frac{J^4}{2J+1}.
\end{eqnarray}
Here $x_{\rm em}$ is the abundance of the emitting species per H nucleus, $T$ is the gas temperature, $Z(T)$ is the partition function, $A_{J,J-1}$ is the Einstein $A$ coefficient from transitions from state $J$ to state $J-1$, $E_J$ is the energy of state $J$, $B$ is the rotation constant for the emitting molecule, and $\mu$ is the electric dipole moment of the emitting molecule. The first equation is simply the statement that the energy loss rate is given by the abundance of emitters multiplied by the fraction of emitters in the $J$ state in question times the spontaneous emission rate for this state times the energy emitted per transition. Note that there is no explicit density dependence as a result of our assumption that the level with which we are concerned is in LTE. The latter two equations are general results for quantum rotors.

The CO molecule has $B=57$ GHz and $\mu=0.112$ Debye, and at Solar metallicity its abundance in regions where CO dominates the carbon budget is $x_{\rm CO} \approx 1.1\times 10^{-4}$. Plugging in these two values, and evaluating for $J$ in the range $2-5$, typical cooling rates are of order $10^{-27}-10^{-26}$ erg cm$^{-3}$ when the temperature is $\sim 10$ K. This is why the equilibrium temperatures of molecular clouds are $\sim 10$ K.

\subsection{Implications}

The calculation we have just performed has two critical implications that strongly affect the dynamics of molecular clouds. First, the temperature will be relatively insensitive to variations in the local heating rate. The cosmic ray and photoelectric heating rates are to good approximation temperature-independent, but the cooling rate is extremely temperature sensitive because, for the dominant cooling lines of CO have level energies are large compared to $k_B T$. Examining equation (\ref{eq:lambdaco}) would seem to suggest that the cooling rate is exponentially sensitive to temperature. In practice the sensitivity is not quite that great, because which $J$ dominates changes with temperature, but numerical calculations still show that $\Lambda_{\rm CO}$ varies with $T$ to a power of $p \sim 2-3$. This means that a factor $f$ increase in the local heating rate will only change the temperature by a factor $\sim f^{1/p}$. Thus we expect molecular clouds to be pretty close to isothermal, except near extremely strong local heating sources.

A second important point is the timescales involved. The gas thermal energy per H nucleus is
\begin{equation}
e \approx \frac{1}{2}\left(\frac{3}{2}k T\right) = 10^{-15} \left(\frac{T}{10\mbox{ K}}\right)\mbox{ erg}
\end{equation}
The factor of $1/2$ comes from 2 H nuclei per H$_2$ molecule, and the equation is only approximate because this neglects quantum mechanical effects that are non-negligible at these low temperatures. However, a correct accounting for these only leads to order unity changes in the result. 

The characteristic cooling time is $t_{\rm cool} = e/\Lambda_{\rm CO}$. Suppose we have gas that is mildly out of equilibrium, say $T=20$ K instead of $T=10$ K. The heating and cooling are far out of balance, so we can ignore heating completely compared to cooling. At the cooling rate of $\Lambda_{\rm CO}=4\times 10^{-26}$ erg s$^{-1}$ for 20 K gas, $t_{\rm cool} = 1.6$ kyr. In contrast, the crossing time for a molecular cloud is $t_{\rm cr} = L/\sigma \sim 7$ Myr for $L=30$ pc and $\sigma = 4$ km s$^{-1}$. The conclusion of this analysis is that radiative effects happen on time scales {\it much} shorter than mechanical ones. Mechanical effects, such as the heating caused by shocks, simply cannot push the gas any significant way out of radiative equilibrium.
