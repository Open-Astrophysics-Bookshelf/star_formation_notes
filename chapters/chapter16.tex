\chapter{Protostar Formation}
\label{ch:protostar_form}

\marginnote{
\textbf{Suggested background reading:}
\begin{itemize}
\item \href{http://adsabs.harvard.edu/abs/2014prpl.conf..173L}{Dunham, M.~M., et al. 2014, in "Protostars and Planets VI", ed.~H.~Beuther et al., pp.~195-218}, sections 1-4 \nocite{dunham14a}
\end{itemize}
\textbf{Suggested literature:}
\begin{itemize}
\item \href{http://adsabs.harvard.edu/abs/2013ApJ...763....6T}{Tomida, K., et al., 2013, ApJ, 763, 6} \nocite{tomida13a}
\end{itemize}
}

The next two chapters focus on the structure and evolution of protostars. Our goal will be to understand when and why collapse stops, leading to formation of a pressure-supported object, and how those objects subsequently evolve into main sequence stars. This chapter focuses on the dynamics and thermal behavior of the material at the center of a collapsing core as it settles into something we can describe as a star, and on the structure of the envelope around this protostar. Chapter \ref{ch:protostar_evol} is focused on the evolution of this object, both internally and in its appearance on the HR diagram. 

\section{Thermodynamics of a Collapsing Core}

We will begin by considering what happens at the center of a collapsing core where the density is rising rapidly as material collapses. 

\subsection{The Isothermal-Adiabatic Transition}
\label{ssec:iso_adiabat}

Thus far we have treated the gas in star-forming regions as approximately isothermal, but this assumption must break down at some point. At low density there are minor deviations from isothermality that result from the density dependence of various heating and cooling processes, but these are fairly minor, in the sense that they are unable to significantly impede collapse. For example, the proposed \citet{larson05a} EOS discussed in chapter \ref{ch:imf_th} only gets as stiff as $T\propto \rho^{0.07}$ at high density, corresponding to a polytrope $P\propto \rho^{\gamma}$ with $\gamma=1.07$. Spherical objects can only be stable if $\gamma>4/3$, so gas with $\gamma=1.07$ is still in the unstable regime. In contrast, if the gas is not able to radiate at all, it will behave adiabatically. This means it will approach a polytrope with $\gamma=7/5$ or $5/3$, depending on whether the gas temperature is high enough to excite the rotational and vibrational levels of H$_2$ or not.\footnote{In actuality the value of $\gamma$ for H$_2$ is more complicated than that, but this detail is unimportant for our purposes.} Either of these values is $>4/3$, and thus sufficient to halt collapse.

Let us make some estimates of when deviations from isothermality that are significant enough to slow collapse will occur. Since we are dealing with the collapse of the first region to fall in, we can probably safely assume that this material has very low angular momentum and treat the collapse as spherical -- higher angular momentum material will only fall in later, since removal of angular momentum by the disk takes a while. The behavior of this material has been studied by a number of authors, going all the way back to \citet{larson69a}, but the treatment here follows that of \citet{masunaga98a} and \citet{masunaga00a}.

At high densities inside a core immediately before a central star forms and begins to radiate, the dominant source of energy is adiabatic compression of the gas. Let $e$ be the thermal energy per unit mass of a particular gas parcel, and let $\Gamma$ and $\Lambda$ be the rates of change in $e$ due to heating and cooling processes, i.e.,
\begin{equation}
\frac{de}{dt} = \Gamma - \Lambda.
\end{equation}
As the gas collapses it will heat up due to adiabatic compression. The first law of thermodynamics tells us that the heating rate due to this process is
\begin{equation}
\label{eq:gamma_ad}
\Gamma = -p \frac{d}{dt}\left(\frac{1}{\rho}\right),
\end{equation}
where $\rho$ and $p=\rho c_s^2$ are the gas density and pressure, and $c_s$ is the isothermal sound speed. Since $1/\rho$ is the specific volume, meaning the volume per unit mass occupied by the gas, this term is just $p\,dV$, the work done on the gas in compressing it. If the gas is collapsing in free-fall, the compression time scale is about the free-fall timescale $t_{\rm ff} = \sqrt{3\pi/32G\rho}$, so we expect
\begin{equation}
\Gamma = C_1 c_s^2 \sqrt{4\pi G\rho},
\end{equation}
where $C_1$ is a constant of order unity that will depend on the exact collapse solution, and the factor of $\sqrt{4\pi}$ has been inserted for future convenience.

The main cooling source is thermal emission by dust grains, which at the high densities with which we are concerned are thermally very well coupled to the gas. Let us first consider the case where the gas is optically thin to this thermal radiation, so the cooling rate per unit mass is simply given by the rate of thermal emission,
\begin{equation}
\Lambda_{\rm thin} = 4\kappa_{\rm P} \sigma_{\rm SB} T^4.
\end{equation}
Here $\sigma_{\rm SB}$ is the Stefan-Boltzmann constant and $\kappa_{\rm P}$ is the Planck mean specific opacity of the gas-dust mixture. As long as $\Lambda \gtrsim \Gamma$, the gas will remain isothermal. (Strictly speaking if $\Lambda > \Gamma$ the gas will cool, but that is because we have left out other sources of heating, such as cosmic rays and the fact that the gas and dust are bathed in a background IR radiation field from other stars.) If we equate the heating and cooling rates, using for $T$ the temperature in the isothermal gas, we therefore will obtain a characteristic density beyond which the gas can no longer remain isothermal. Doing so gives
\begin{eqnarray}
\rho_{\rm thin} & = & \frac{4}{\pi} \frac{\kappa_{\rm P}^2 \sigma_{\rm SB}^2 \mu^2 m_{\rm H}^2 T^6}{C_1^2 G k_B^2} \\
& = & 5\times 10^{-15} \mbox{ g cm}^{-3}\, C_1^{-2} \kappa_{\rm P,-2}^2 T_1^6
\end{eqnarray}
where $\mu$ is the mean mass per particle in units of $m_{\rm H}$ ($\mu=2.3$ for fully molecular gas), and we have set $c_s = \sqrt{k_B T/\mu m_{\rm H}}$. In the second line, $T_1 = T/10$ K and $\kappa_{\rm P,-2} = \kappa_{\rm P}/0.01$ cm$^2$ g$^{-1}$, a typical value for thermal radiation at a temperature of $\approx 10$ K and Milky Way dust grains. Thus we find that compressional heating and optically thin cooling to balance at about $10^{-14}$ g cm$^{-3}$.

A second important density is the one at which the gas starts to become optically thick to its own re-emitted infrared radiation. Suppose that the optically thick region at the center of our core has some mean density $\rho$ and radius $R$. The condition that the optical depth across it be unity then reduces to
\begin{equation}
\label{eq:thick_thin}
2 \kappa_{\rm P} \rho R \approx 1.
\end{equation}
If this central region corresponds to the size of the region that is no longer in free-fall collapse and is instead thermally supported, then its size must be comparable to the Jeans length at its lowest temperature, i.e., $R\sim \lambda_J = \sqrt{\pi c_s^2/(G\rho)}$. Thus we set
\begin{equation}
R = C_2 \frac{2\pi c_s}{\sqrt{4\pi G \rho}},
\end{equation}
where $C_2$ is again a constant of order unity, and Masunaga et al.\ find based on numerical results that $C_2\approx 0.75$. Plugging this value of $R$ into equation (\ref{eq:thick_thin}), we obtain the characteristic density at which the gas transitions from optically thin to optically thick,
\begin{eqnarray}
\rho_{\tau\sim 1} & = & \frac{1}{4\pi} C_2^{-2} \frac{\mu m_{\rm H} G}{\kappa_{\rm P}^2 k_B T} \\
& = & 
1.5\times 10^{-13}\mbox{ g cm}^{-3}\,  C_2^{-2} \kappa_{\rm P,-2}^{-2} T_1^{-1}
\end{eqnarray}
This is not very different from the value for $\rho_{\rm thin}$, so in general for reasonable collapse conditions we expect that cores transition from isothermal to close to adiabatic at a density of $\sim 10^{-13}-10^{-14}$ g cm$^{-3}$.

It is worth noting that ratio of $\rho_{\rm thin}$ to $\rho_{\tau\sim 1}$ depends extremely strongly on both $\kappa_{\rm P}$ (to the 4th power) and $T$ (to the 7th), so any small change in either can render them very different. For example, if the metallicity is super-solar then $\kappa_{\rm P}$ will be larger, which will increase $\rho_{\rm thin}$ and decrease $\rho_{\tau\sim 1}$.
Similarly, if the region is somewhat warmer, for example due to the presence of nearby massive stars, then $\rho_{\rm thin}$ will increase and $\rho_{\tau\sim 1}$ will decrease.

If $\rho_{\tau\sim 1} < \rho_{\rm thin}$, the collapsing gas will become optically thick before heating becomes faster than optically thin cooling. In this case we must compare the heating rate due to compression with the cooling rate due to optically thick cooling instead of optically thin cooling. The cooling rate for an optically thick region is determined by how quickly radiation can diffuse out. If we have a central region of optical depth $\tau\gg 1$, the effective speed of the radiation moving through it is $c/\tau$, so the time required for the radiation to diffuse out is
\begin{equation}
t_{\rm diff} = \frac{l\tau}{c} = \frac{\kappa_{\rm P} \rho l^2}{c}
\end{equation}
where $l$ is the characteristic size of the core. Inside the optically thick region matter and radiation are in thermal balance, so the radiation energy density approaches the blackbody value $a_R T^4$. The radiation energy per unit mass is therefore $a_R T^4/\rho$. Putting all this together, and taking $l=2 R$ as we did before in computing $\rho_{\tau\sim 1}$, the optically thick cooling rate per unit mass is
\begin{equation}
\Lambda_{\rm thick} = \frac{a_R T^4/\rho}{t_{\rm diff}} = \frac{\sigma_{\rm SB} T^4}{\kappa_{\rm P} \rho^2 R^2},
\end{equation}
where $\sigma_{\rm SB} =ca_R /4$. If we equate $\Lambda_{\rm thick}$ and $\Gamma$, we get the characteristic density where the gas becomes non-isothermal in the optically thick regime
\begin{eqnarray}
\rho_{\rm thick} & = & \left(\frac{C_1^2 G \sigma_{\rm SB}^2 \mu^4 m_{\rm H}^4 T^4}{4\pi^3 C_2^4 k_B^4 \kappa_{\rm P}^2}\right)^{1/3} \\
& = & 5\times 10^{-14}\mbox{ g cm}^{-3}\, \frac{C_1^{2/3}}{C_2^{4/3}} \kappa_{\rm P,-2}^{-2/3} T_1^{4/3}.
\end{eqnarray}
This is much more weakly dependent on $\kappa_{\rm P}$ and $T$, so we can now make the somewhat more general statement that, even for supersolar metallicity or warmer regions, we expect a transition from isothermal to adiabatic behavior somewhere in the vicinity of $10^{-14}-10^{-13}$ g cm$^{-3}$.

\subsection{The First Core}

The transition to an adiabatic equation of state, with $\gamma>4/3$, means that the collapse must at least temporarily halt. The result will be a hydrostatic object that is supported by its own internal pressure. This object is known as the first core, or sometimes a Larson's first core, after Richard Larson, who first predicted this phenomenon.

We can model the first core reasonably well as a simple polytrope, with index $n$ defined by $n=1/(\gamma-1)$. When the temperature in the first core is low, $\gamma\approx 5/3$ and $n\approx 3/2$, and for a more massive, warmer core $\gamma \approx 7/5$ ($n\approx 5/2$). The theory of polytropes can be found in many standard stellar structure textbooks \citep[e.g.,][]{chandrasekhar39a, kippenhahn94a}, and so we will not rehearse the topic here, and will simply quote the result. For a polytrope of central density $\rho_c$, the radius and mass are
\begin{eqnarray}
R & = & a \xi_1\\
M & = & -4\pi a^3 \rho_c \left(\xi^2\frac{d\theta}{d\xi}\right)_1,
\end{eqnarray}
where $\xi=r/a$ is the dimensionless radius, $\theta = (\rho/\rho_c)^{1/n}$ is the dimensionless density, the subscript $1$ refers to the value at the edge of the sphere (where $\theta=0$), the factors $\xi_1$ and $(\xi\, d\theta/d\xi)_1$ can be determined by integrating the Lane-Emden equation, and the scale factor $a$ is defined by
\begin{equation}
a^2 = \frac{(n+1) K}{4\pi G} \rho_c^{\frac{1-n}{n}}.
\end{equation}
The factor $K = p/\rho^{\gamma}$ is the polytropic constant, which is determined by the specific entropy of the gas.

For our first core, the specific entropy will just be determined by the density at which the gas transitions from isothermal to adiabatic. If we let $\rho_{\rm ad}$ be the density at which the gas becomes adiabatic, then the pressure at this density is $p = \rho_{\rm ad} c_{s0}^2$, where $c_{s0}$ is the sound speed in the isothermal phase, and $K = c_{s0}^2 \rho_{\rm ad}^{1-\gamma}$. For $\gamma=5/3$ ($n=1.5$) we have $\xi_1=3.65$ and $(\xi^2\, d\theta/d\xi)_1=-2.71$, and plugging in we get
\begin{eqnarray}
R & = & 2.2\mbox{ AU} \; T_1^{1/2} \rho_{c,-10}^{1/6} \rho_{\rm ad,-13}^{-1/3} \\
M & = & 0.059\,\msun\; T_1^{1/2} \rho_{c,-10}^{7/6} \rho_{\rm ad,-13}^{-1/3},
\end{eqnarray}
where $\rho_{c,-10} = \rho_c / 10^{-10}$ g cm$^{-3}$ and $\rho_{\rm ad,-13} = \rho_{\rm ad}/10^{-13}$ g cm$^{-3}$. Our decision to scale $\rho_c$ to $10^{-10}$ g cm$^{-3}$ will be justified in a moment. Repeating the exercise for $\gamma=7/5$ ($n=2.5$) gives almost identical results, with slightly different leading constants. We therefore conclude that the first core is an object a few AU in size, with a mass of a few hundredths of a Solar mass.

\subsection{Second Collapse}

The first core is a very short-lived phase in the evolution of the protostar. To see why, let us estimate its temperature. The temperature inside the sphere rises as $T\propto \rho^{\gamma-1}$, so the central temperature is
\begin{equation}
T_c = T_0 \left(\frac{\rho_c}{\rho_{\rm ad}}\right)^{\gamma-1},
\end{equation}
where $T_0$ is the temperature in the isothermal phase. Thus the central temperature will be higher than the boundary temperature by a factor that is determined by how high the central density has risen, which in turn will be determined by the amount of mass that has accumulated on the core.

In general we have $M\propto \rho_c^{(3+n)/(2n)}$, or $M\propto \rho_c^{(3\gamma-2)/2}$. We also have $T_c\propto \rho_c^{\gamma-1}$. Combining these results, we have
\begin{equation}
T_c \propto M^{(2\gamma-2)/(3\gamma-2)}.
\end{equation}
The exponent is $0.44$ for $\gamma=5/3$ and $0.36$ for $\gamma=7/5$. Plugging in some numbers, $M = 0.06\msun$, $\rho_{\rm ad}=10^{-13}$ g cm$^{-3}$, and $\gamma=5/3$ gives $\rho_c = 10^{-10}$ g cm$^{-3}$ and $T_c=1000$ K. Thus we see that by the time anything like $0.1$ $\msun$ of material has accumulated on the first core, compression will have caused its central temperature to rise to $1000$ K or more.

This causes yet another change in the thermodynamics of the gas, because all the hydrogen is still molecular, and molecular hydrogen has a binding energy of $4.5$ eV. In comparison, the kinetic energy per molecule for molecular hydrogen at a temperature $T$ is $(3/2) k_B T = 0.13 T_3$ eV, where $T_3=T/(1000\mbox{ K})$. At 1000 K this means that the mean molecule still has only a few percent of the kinetic energy that would be required to dissociate it. However, there is a non-negligible tail of the Maxwellian distribution that is moving fast enough for collisions to produce dissociation. Each of these dissociative collisions removes $4.5$ eV from the kinetic energy budget of the gas and puts it into chemical energy instead. Since dissociations are occurring on the tail of the Maxwellian, any slight increase in the temperature dramatically increases the dissociation rate, moving even more kinetic energy into chemical energy.

This effectively acts as a thermostat for the gas, in much the same way that a boiling pot of water stays near the boiling temperature of water even when energy is added, because all the extra energy that is provided goes into changing the chemical phase of the water rather than raising its temperature. Detailed numerical calculations of this effect show that at temperatures above $1000-2000$ K, the equation of state becomes closer to $T\propto \rho^{0.1}$, or $\gamma=1.1$. This is again below the critical value of $\gamma=4/3$ required to have a hydrostatic object, and as a result the center of the first core again goes into something like free-fall collapse.

This is called the second collapse. The time required for it is set by the free-fall time at the central density of the first core, which is only a few years. This collapse continues until all the hydrogen dissociates. The hydrogen also ionizes during this collapse, since the ionization potential of $13.6$ eV is not very different from the dissociation potential of $4.5$ eV. Only once all the hydrogen is dissociated and ionized can a new hydrostatic object form. At this point the gas is warmer than $\sim 10^4$ K, is fully ionized, and the new hydrostatic object is a true protostar. It is supported by degeneracy pressure at first when its mass is low, and then as more mass arrives it heats up and becomes supported by thermal pressure. 

An important point to make here is that this discussion implies that brown dwarfs, at least those of sufficiently low mass, do not undergo a prompt second collapse. Instead, their first cores never accumulate enough mass to dissociate the molecules at their center. This is not to say that dissociation never happens in them, and that second collapse never occurs. A brown dwarf-mass first core will still radiate from its surface and, lacking any internal energy source, this energy loss will have to be balanced by compression. As the gas compresses the temperature and entropy will rise, and, if the object does not become supported by degeneracy pressure first, the central temperature will eventually rise enough to produce second collapse. The difference for a brown dwarf is that this will only occur once slow radiative losses cause a temperature rise, which may take a very long time compared to formation. For stars, in contrast, there is enough mass to reach the critical temperature by compression during formation.

\section{The Protostellar Envelope}

Once a protostar is born at the center of a collapsing cloud, we can ask both about the structure immediately around it and about its internal structure. We defer the latter to Chapter \ref{ch:protostar_evol}, and focus here on the envelope around the newborn protostar.

\subsection{Accretion Luminosity}

The temperature of the gas around the newborn protostar is determined by the radiation that the central star emits. At early times the star has not reached the main sequence or ignited any nuclear burning, so gravity is the only important energy source in the problem. Even if nuclear burning does start, we will see that it is negligible for low mass stars. The protostar is a hydrostatic object, although it undergoes secular contraction, so that gas striking its surface comes to a halt in an accretion shock. In this shock its kinetic energy is converted to heat, which is then radiated away.

The detailed structure of the accretion shock was first worked out by \citet{stahler80a, stahler80b}. The summary is that the energy radiated away at the shock is roughly
\begin{equation}
L_{\rm acc} = \frac{G M_* \dot{M}_*}{R_*},
\end{equation}
where $M_*$, $\dot{M}_*$, and $R_*$ are the mass, accretion rate, and radius for the protostar. We will see in Chapter \ref{ch:protostar_evol} that $R_*$ is typically a few $\rsun$ (and indeed this is consistent with the observed radii of T Tauri stars). We have previously calculated typical accretion rates of $\dot{M}_*\sim 10^{-5}$ $\msun$ yr$^{-1}$ for low mass stars.
Plugging in these numbers, we find
\begin{equation}
L_{\rm acc} = 30\lsun\, \dot{M}_{*,-5} M_{*,0} R_{*,1}^{-1},
\end{equation}
where $\dot{M}_{*,-5}=\dot{M}_*/(10^{-5}\msun\mbox{ yr}^{-1})$, $M_{*,0}=M_*/\msun$, and $R_{*,1}=R_*/(10\rsun)$.
Thus a typical low mass protostar can easily put out many tens of $\lsun$ in accretion power, far greater than what it would produce from nuclear burning on the main sequence. 

We can also estimate the effective temperature of the stellar surface due to accretion. The infalling gas arrives in free-fall at a velocity 
\begin{equation}
v_{\rm ff}=\sqrt{\frac{2GM_*}{R_*}} = 200\mbox{ km s}^{-1} \, M_{*,0}^{1/2} R_{*,1}^{-1/2}.
\end{equation}
The vastly exceeds the sound speed of a few km s$^{-1}$ in gas at a temperature of $\sim 10^3-10^4$ K, so the gas must decelerate in a strong shock with a Mach number of order 100. For a strong shock, one where the Mach number $\mathcal{M} \gg 1$, the Rankine-Hugoniot jump conditions tell us that the post-shock temperature is
\begin{equation}
T_2 = \frac{2\gamma(\gamma-1)}{(\gamma+1)^2} \mathcal{M}^2 T_1
= \frac{2\gamma(\gamma-1)}{(\gamma+1)^2} \frac{v_{\rm shock}^2}{c_1^2} T_1
= \frac{2(\gamma-1)}{(\gamma+1)^2} \frac{\mu m_{\rm H}}{k_B} v_{\rm shock}^2,
\end{equation}
where $c_1$ is the adiabatic sound speed in the pre-shock gas.

Taking $v_{\rm shock}=v_{\rm ff}$, $\gamma=5/3$ for a monatomic gas, and $\mu=1.4$ for the pre-shock gas (assuming it to be neutral hydrogen), and plugging in we get
\begin{equation}
T_2 = 1.2\times 10^6 \, M_{*,0} R_{*,1}^{-1} \mbox{ K}.
\end{equation}
In other words, the post-shock gas is heated to temperatures such that it emits in UV and x-rays. The incoming gas will be extremely opaque to this radiation due to the opacity provided by both free electrons and numerous lines of multiply ionized metal atoms such as iron. As a result all the radiation emitted by the post-shock gas will be absorbed in a small region immediately outside the shock and reprocessed until it becomes blackbody emission. The stellar surface therefore emits as a blackbody, whose temperature we can calculate in the standard way:
\begin{eqnarray}
L_{\rm acc} & = & 4\pi R_*^2 \sigma_{\rm SB} T_*^4 \\
T_* & = & 4300 \dot{M}_{*,-5}^{1/4} M_{*,0}^{1/4} R_{*,1}^{-3/4} \mbox{ K}.
\end{eqnarray}
Thus the star is effectively a blackbody at a surface temperature comparable to that of a main sequence star.

\subsection{The Dust Destruction Front}

Now let us consider the effect of this luminosity on the gas around the protostar. Consider a spherical black dust grain of radius $a$ some distance $r$ from the star. It absorbs radiation at a rate
\begin{equation}
\Gamma = \frac{L_{\rm acc}}{4\pi r^2} \pi a^2 = \pi a^2 \sigma_{\rm SB} T_*^4 \left(\frac{R_*}{r}\right)^2
\end{equation}
and radiates it at a rate\footnote{This expression is only valid if the wavelengths characteristic of the peak of the blackbody curve at temperature $T_d$ are small compared to the circumference of the grain. For the dust temperature of $\approx 1000$ K we will insert below, this implies that the result is valid for grains with characteristic sizes $\gtrsim 1$ $\mu$m. Smaller grains will have lower values of $\Lambda$ and thus higher equilibrium temperatures.}
\begin{equation}
\Lambda = 4\pi a^2 \sigma_{\rm SB} T_d^4,
\end{equation}
where $T_d$ is the dust grain's temperature. Equating these two, the temperature of the grain is
\begin{equation}
T_d = \left(\frac{R_*}{2r}\right)^{1/2} T_*
\end{equation}

Even the most refractory materials out of which interstellar dust is made, such as graphite and silicate, will vaporize at temperatures larger than $\sim 1000-1500$ K. The exact temperature depends on the chemical composition of the grains. Thus when $r/R_*$ is too small grains cannot survive. They are vaporized. We therefore expect the protostar to be surrounded by dust-free region.

Since the ionizing radiation produced at the shock at the stellar surface all gets absorbed close to the shock, and the star is shining into this dust-free region as a blackbody at a temperature of only a few thousand K, the gas in this region is primarily neutral. Neutral atomic gas with no dust in it is essentially transparent to visible radiation, so in this region the opacity is tiny, and stellar radiation is able to free-stream outward. The dust-free neutral region is called the opacity gap.

As one moves away from the star the equilibrium grain temperature drops, and eventually one reaches a surface where dust grains can exist. This is called the dust destruction radius, since incoming gas that reaches this radius has its grains destroyed. If we plug the grain destruction temperature into our equation for $T_d$, we can solve for the dust destruction radius:
\begin{equation}
r_d = \frac{R_*}{2} \left(\frac{T_*}{T_d}\right)^2 = 0.4 \,  T_{d,3}^{-2} \dot{M}_{*,-5}^{1/2} M_{*,0}^{1/2} R_{*,1}^{-1/2} \mbox{ AU},
\end{equation}
where $T_{d,3}=T_d/(1000\mbox{ K})$ is the dust destruction temperature in units of 1000 K. Thus the dust-free region extends to $\sim 1$ AU around an accreting protostar.

\subsection{Temperature Structure and Observable Properties}

Now let us consider the material beyond the dust destruction front. At the front the gas density is given roughly by the condition
\begin{eqnarray}
\dot{M}_* & = & 4\pi r_d^2 \rho v_{\rm ff} \\
\rho & = & \frac{\dot{M}_*}{\sqrt{8 \pi^2 G M_* r_d^3}} \\
& = & 4\times 10^{-13} \, \dot{M}_{*,-5}^{1/4} M_{*,0}^{-7/4} R_{*,1}^{3/4} T_{d,3}^3\mbox{ g cm}^{-3}.
\end{eqnarray}
Just inside the front, the stellar spectrum is nearly a blackbody at a temperature of a few thousand Kelvin (4300 K above), so the peak wavelength from Wien's Displacement Law is
\begin{equation}
\lambda \approx \frac{2989 \mu\mathcal{m}}{T} = 675 \, \dot{M}_{*,-5}^{-1/4} M_{*,0}^{-1/4} R_{*,1}^{3/4}\mbox{ nm},
\end{equation}
placing it in the visible.

The opacity of gas with Milky Way dust composition at 675 nm is roughly $\kappa=3000$ cm$^2$ g$^{-1}$, so the mean free-path of a stellar photon moving through the dust destruction front is $(\kappa\rho)^{-1} \approx 10^8$ cm ($\sim R_\earth$). This is a tiny length scale compared to any other scale in the problem, such as the size of the core, the size of the opacity gap, or even the radius of the protostar. Thus all the starlight that strikes the dust destruction front will immediately be absorbed by the dust grains. They will re-emit it as thermal radiation with a peak wavelength determined by their blackbody temperature, which will be a factor of $\sim 4$ lower than the stellar surface temperature. At around 1.8 $\mu$m, a factor of 3 longer wavelength than the 675 nm we started with, the opacity is drops to around 1000 cm$^2$ g$^{-1}$, so the mean free path is a factor of 3 larger. Nonetheless, this is still tiny, so all the re-emitted radiation will also be absorbed.

Since we are in a situation where all the radiation is absorbed and re-emitted many times, it is reasonable to treat this as a diffusion problem. Protostellar radiation free-streams from the surface, through the opacity gap, and is absorbed and thermalized at the dust destruction front. Then it must diffuse out through the dust envelope. This is essentially the same calculation that is made for radiation diffusing outward through a star, and the equation describing it is the same:
\begin{equation}
F = -\frac{c}{3\rho \kappa_R} \nabla E,
\end{equation}
where $F$ is the radiation flux, $E$ is the radiation energy density, and $\kappa_R$ here is the Rosseland mean opacity, meaning the mean of the frequency-dependent opacity using a weighting function that is equal to the temperature derivative of the Planck function. Note here that $\kappa_R$ is a function of $T$.

The repeated absorption and re-emission of radiation forces it into thermal equilibrium with the gas, so $E$ is simply the energy density of a thermal radiation field at the gas temperature: $E=a_R T^4$. Since no energy is added or removed from the radiation field as it diffuses outward through the envelope, $F=L_{\rm acc}/(4\pi r^2)$. Putting this together, we have
\begin{equation}
L_{\rm acc}=-\frac{16\pi ca_R r^2}{3\rho\kappa_R} T^3 \frac{dT}{dr}
\end{equation}

For a given density structure and a model of dust grains that specifies $\kappa_R(T)$, this equation allows us to estimate the temperature structure in the protostellar envelope. For reasonable grain models we expect $\kappa_R\propto T^\alpha$ with $\alpha\approx 0.8$ in the temperature range of a few hundred K. Let us suppose that the density distribution in the envelope looks something like a powerlaw, so $\rho\propto r^{-k_{\rho}}$. Finally, let us also suppose that the temperature also behaves like a powerlaw in radius, $T\propto r^{-k_T}$. The left hand side of the equation is a constant, and we have now worked out how the right hand side varies with $r$. Plugging in all the radial dependences on the RHS, and knowing that they must sum to zero since the LHS is a constant, we get
\begin{equation}
k_T = \frac{k_{\rho} + 1}{4-\alpha}.
\end{equation}

Thus in the freely-falling part of the envelope, where $k_{\rho}\approx 3/2$, we have $k_T\approx 0.8$. In our fiducial example, where the temperature is 1000 K at 0.4 AU, we would expect the temperature to drop to 300 K at around 2 AU, to 100 K at around 8 AU, and back to the background temperature of 10 K at around 150 AU. In the outer part of the envelope the falloff in temperature can be either steeper or shallower depending on how the density falls off -- sharper density falloffs (larger $k_{\rho}$) lead to sharper temperature falls (larger $k_T$) as well.

Of course this approximation only applies as long as the radiation is trapped by the dust, and the dust opacity is highest for high frequency radiation. Once the dust temperature falls off to less than $\sim 100$ K, depending on the size of the core, the radiation is free to escape instead. Even further in, where the dust temperature is higher, long wavelength radiation can escape freely.

As a result the spectrum inside the core is never truly a blackbody, since radiation at long wavelengths never reaches thermal equilibrium. The emitted spectrum is also complicated by this behavior. We can think of this as follows: for a star, there is something close to a single well-defined photosphere at all frequencies because the density drops off sharply. For a dust cloud, on the other hand, the density drop is not sharp, and so the photosphere, the surface of optical depth unity (or 2/3 if you prefer) is in different places at different frequencies. At high frequencies it is near the core surface because the opacity is high, and at low frequencies the low opacity allows it to be much farther in. For this reason, centrally-heated cores do not emit as blackbodies.

In order to truly determine the temperature distribution within a core it is necessary to either use a more sophisticated analytic treatment (for example one is given in \citealt{chakrabarti05a}) or to proceed numerically. If one wants a more sophisticated density structure that is not spherical, numerical methods are also required. Of course all of this only applies as long as a great deal of mass remains in the envelope, so that it is optically thick to both the star's direct radiation and to the re-radiated thermal radiation from the dust destruction front. In terms of our evolutionary classes, all of this applies to class 0 and class I sources.
