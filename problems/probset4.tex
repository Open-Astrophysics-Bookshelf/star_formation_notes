\problemset

\begin{enumerate}

\item {\bf A Simple Protostellar Evolution Model.}\\
Consider a protostar forming with a constant accretion rate $\dot{M}$. The accreting gas is fully molecular, arrives at free-fall, and radiates away a luminosity $L_{\rm acc} = f_{\rm acc} G M \dot{M}/R$ at the accretion shock, where $M$ and $R$ are the instantaneous protostellar mass and radius, and $f_{\rm acc}$ is a numerical constant of order unity. At the end of contraction the resulting star is fully ionized, all its deuterium has been burned to hydrogen, and it is in hydrostatic equilibrium. The ionization potential of hydrogen is $\psi_I = 13.6$ eV per amu, the dissociation potential of molecular hydrogen is $\psi_M=2.2$ eV per amu, and the energy released by deuterium burning is $\psi_D\approx 100$ eV per amu of total gas (not per amu of deuterium).
\begin{enumerate}
\item First consider a low-mass protostar whose internal structure is well-described by an $n=3/2$ polytrope. Compute the total energy of the star, including thermal energy, gravitational energy, and the chemical energies associated with ionization, dissociation, and deuterium burning.
\item Use your expression for the total energy to derive an evolution equation for the radius for a star. Assume the star is always on the Hayashi track, which for the purposes of this problem we will approximate as having a fixed effective temperature $T_{\rm H} = 3500$ K.
\item Numerically integrate your equation and plot the radius as a function of mass for $\dot{M} = 10^{-5}$ $\msun$ yr$^{-1}$ and $f_{\rm acc}=3/4$. As an initial condition, use $R=2.5$ $\rsun$ and $M=0.01$ $\msun$, and stop the integration at a mass of $M=1.0$ $\msun$. Plot the radius and luminosity as a function of mass; in the luminosity, include both the the accretion luminosity and the internal luminosity produced by the star.
\item Now consider two modifications we can make to allow the model to work for massive protostars. First, since massive stars are radiative, the polytropic index will be roughly $n=3$ rather than $n=3/2$. Second, the surface temperature will in general be larger than the Hayashi limit, so take the luminosity to be $L=\max[L_{\rm H}, \lsun(M/\msun)^3]$, where $L_{\rm H}=4\pi R^2 \sigma T_{\rm H}^4$ and $R$ is the stellar radius. Modify your evolution equation for the radius to include these effects, and numerically integrate the modified equations up to $M=50$ $\msun$ for $\dot{M} = 10^{-4}$ $\msun$ yr$^{-1}$ and $f_{\rm acc}=3/4$, using the same initial conditions as for the low mass case. Plot $R$ and $L$ versus $M$.
\item Compare your result to the fitting formula for the ZAMS radius of solar-metallicity stars as a function of $M$ in \citet{tout96a}\footnote{\href{http://adsabs.harvard.edu/abs/1996MNRAS.281..257T}{Tout et al., 1996, MNRAS 281, 257}}. Find the mass at which the massive star would join the main sequence. Your plots for $R$ and $L$ are only valid up to this mass, because this simple model does not include hydrogen burning.
\end{enumerate}

\item \textbf{Self-Similar Viscous Disks.}\\
Consider a protostellar disk orbiting a star, governed by the usual viscous evolution equation
\begin{displaymath}
\frac{\partial\Sigma}{\partial t} = \frac{3}{\varpi} \frac{\partial}{\partial \varpi} \left[\varpi^{1/2} \frac{\partial}{\partial \varpi} \left(\nu \Sigma \varpi^{1/2}\right)\right],
\end{displaymath}
where $\Sigma$ is the surface density, $\varpi$ is the radius in cylindrical coordinates, and $\nu$ is the viscosity. Suppose that the viscosity is linearly proportional to the radius, $\nu = \nu_1 (\varpi/\varpi_1)$.
\begin{enumerate}
\item Non-dimensionalize the evolution equation by making a change of variables to the dimensionless position, time, and surface density $x=\varpi/\varpi_1$, $T = t/t_s$, $S = \Sigma/\Sigma_1$, where $t_s = \varpi_1^2/(3\nu_1)$.
\item Use your non-dimensionalized equation to show that
\begin{displaymath}
\Sigma = \left(\frac{C}{3\pi \nu_1}\right) \frac{e^{-x/T}}{x T^{3/2}}
\end{displaymath}
is a solution of the equation for an arbitrary constant $C$.
\item Calculate the total mass in the disk in terms of $C$, $t_s$, and $t$, and calculate the time rate of change of this mass. Based on your result, give a physical interpretation of what the constant $C$ means. (Hint: what units does $C$ have?)
\item Plot $S$ versus $x$ at $T = 1, 1.5, 2$, and $4$. Give a physical interpretation of the results.\\
\end{enumerate}

\item {\bf A Simple T Tauri Disk Model.}\\
In this problem we will construct a simple model of a T Tauri star disk in terms of a few parameters: the midplane density and temperature $\rho_m$ and $T_m$, the surface temperature $T_s$, the angular velocity $\Omega$, and the specific opacity of the disk material $\kappa$. We assume that the disk is very geometrically thin and optically thick, and that it is in thermal and mechanical equilibrium.
\begin{enumerate}
\item Assume that the disk radiates as a blackbody at temperature $T_s$. Show that the surface and midplane temperatures are related approximately by
\begin{displaymath}
T_m \approx \left(\frac{3}{8}\kappa\Sigma\right)^{1/4} T_s,
\end{displaymath}
where $\Sigma$ is the disk surface density.
\item Suppose the disk is characterized by a standard $\alpha$ model, meaning that the viscosity $\nu=\alpha c_s H$, where $H$ is the scale height and $c_s$ is the sound speed. For such a disk the rate per unit area of the disk surface (counting each side separately) at which energy is released by viscous dissipation is $F_d=(9/8) \nu \Sigma \Omega^2$. Derive an estimate for the midplane temperature $T_m$ in terms of $\Sigma$, $\Omega$, and $\alpha$.
\item Calculate the cooling time of the disk in terms of the orbital period. Should the behavior of the disk be closer to isothermal or adiabatic?
\item Consider a disk with a mass of $0.03$ $\msun$ orbiting a $1$ $\msun$ star, which has $\kappa=3$ cm$^2$ g$^{-1}$ and $\alpha=0.01$. The disk runs from 1 to 20 AU, and the surface density varies as $R^{-1}$. Use your model to express $\rho_m$, $T_m$, and $T_s$ as functions of the radius, normalized to 1 AU; i.e., derive results of the form $\rho_m = \rho_0 (r/\mathrm{AU})^p$ for each of the quantities listed. Is your numerical model disk gravitationally unstable (i.e., $Q<1$) anywhere?
\end{enumerate}

\end{enumerate}
